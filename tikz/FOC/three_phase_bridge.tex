% 三相逆变桥
% Author: L.
\documentclass[border=10pt]{standalone}
\usepackage{circuitikz}
\usepackage{siunitx}
\usepackage{amsmath, amssymb}
\begin{document}
\begin{circuitikz}[american voltages, european resistors] \draw
(0,3) node[nigbt] (a_h) {$A_H$}
(0,0) node[nigbt] (a_l) {$A_L$}
(2,3) node[nigbt] (b_h) {$B_H$}
(2,0) node[nigbt] (b_l) {$B_L$}
(4,3) node[nigbt] (c_h) {$C_H$}
(4,0) node[nigbt] (c_l) {$C_L$}

%C
(a_h.C) --++ (-2, 0)node[circ] {} --++
(0,-1) to[C=$C$] (-2, 0)

%Vpp
 (-4, 0) to[V=$V_{dc}$] (-4, 3) 
 (a_h.C) ++ (-2,0) --++ (-2,0) -- (-4,3)
 
%C								
(a_l.E) --++ (-2, 0)node[circ] {} -- (-2, 0)

%Vpp
(a_l.E) ++(-2, 0) --++ (-2, 0) -- (-4, 0) 

%igbt
(a_h.E) -- (a_l.C)
(b_h.E) -- (b_l.C)
(c_h.E) -- (c_l.C)

(a_h.C) node[circ] {} -- (b_h.C) node[circ] {} -- (c_h.C)
(a_l.E) node[circ] {} -- (b_l.E) node[circ] {} -- (c_l.E)


%R
 (1, -2) to[R=$R_b$] (1, -4) node[circ] {}
 (1, -4) ++(210:2) to[R=$R_a$] (1, -4)
 (1, -4) ++(-30:2) to[R=$R_c$] (1, -4)

(a_l.C) ++ (0, .75) node[circ] {} node[anchor=east] {$V_a$}
 --++ (.5, 0) --++(0, -4.5) --++ (210:2.08)  --++ (-60:1.14)

(b_l.C) ++ (0, .75) node[circ] {} node[anchor=east] {$V_b$} 
--++ (.5, 0) --++(0, -3) -| (1,-2)

 (1,-4) ++ (-30:2) -| (4.5,1.5) --++ (-.5, 0) node[circ] {} 
 node[anchor=east] {$V_c$}

;\end{circuitikz}
\end{document}
